\section{Considerations}

\subsection{Limitation of dataset size}
The implemented solution is unable to handle new data points when the
registers are full. This is due to two reasons:

\begin{itemize}
	\item The hardware algorithm, especially the parallelized algorithm, is
		reliant on a given size of its registers, as this also
		determines the size of the multiplexers and the number of
		subtraction elements required. It must be noted that a more
		generalized and sophisticated implementation may be able to
		handle the arrival of new data points.

	\item Due to the quadratic nature of the underling mathematical problem,
		this solution will always exceed the available storage space.
		Depending on the situation, a segmentation of the data and
		therefore partial interpolation might mitigate this limitation.
\end{itemize}


\subsection{Hardware Improvements}

Following hardware optimizations can increase the speed of the implemented
algorithm:

\begin{itemize}
	\item \textbf{Signed Digits:} Using the signed digit notation can speed
		up the this implementation as it heavily relies on
		\textit{Addition/Subtraction.} The duration of these operations
		is then no longer proportional to the bus width.\footnote{Israel
		Koren, Computer Arithmetic Algorithms, 2.3 Signed-Digit Number
		Systems}

	\item \textbf{Booth's Algorithm:} The little amount of multiplication
		needed by the algorithm can be speed up using the Booth's
		algorithm to reduce the number of partial products. This
		algorithm itself heavily relies on
		\textit{Addition/Subtraction}. Again, the signed digit notation
		would be welcomed in this case. Though it must be mentioned that
		the compatibility of Booth's algorithm with the signed digit
		notation was not researched. It is unclear that the calculation
		scheme would still hold for the signed digit notation.
		\footnote{Israel Koren, Computer Arithmetic Algorithms,
		6.1 Reducing the Number of Partial Products}
\end{itemize}
