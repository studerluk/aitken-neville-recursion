\section{Methodology of Interpolation\footnote{This section is a summary of the
	Script \textit{Newton Polynomial Interpolation (Collocation)} of the
	Master lecture \textit{Numerical Analysis}}}

\subsection{The collocation problem}
There exists a vast variety of possible interpolation methods. One such method
is through collocation with a high-order polynomial, such as:

\begin{align}
	y(x) &= p(x) = c_0 + c_1 x^1 + c_2 x^2 + \dots
		+ c_{m-1} x^{m-1} + c_m x^m
			&(c_0, c_1, \dots c_m \in \mathbb{R}) \\
	\Rightarrow y(x_k) &= p(x_k) = y_k &(k = 0,1,2,3,\dots,n)
\end{align}

\textit{\textbf{Note:} The degree $m$ should be minimal, but the collocation
	conditions must be met.}

\begin{defn}
	Collocation: \textit{A single curve passing through all corresponding
		measurements.}
\end{defn}

A unique set of coefficients $c_0, c_1, c_2, \dots, c_m$ can be found using
elementary linear algebra if $m=n$. Though, this is proofs to be a very
inefficient method. Furthermore, this method is subject to significant numeric
instabilities.


\subsection{Aitken-Neville recursion}
An alternative method for finding the coefficients to the interpolating
polynomial is the Aitken-Neville recursion. Hereby the data is partitioned.
The \textit{global} polynomial terms are then determined using the relations
between different \textit{parial} polynomials.

\begin{defn}
	Aitken-Neville recursion: \textit{Finding the global polynomial through
		repetitive combination of ever smaller, partial polynomials.}
\end{defn}

This recursive algorithm can be represented in a tabular fashion.
\begin{table}[H]
	\centering
	\begin{tabular}{ r | l l l l l }
		$x_0$	& $y_0 = p_0$	&		&		&		& \\
		$x_1$	& $y_1 = p_1$	& $p_{01}$	&		&		& \\
		$x_2$	& $y_2 = p_2$	& $p_{12}$	& $p_{012}$	&		& \\
		$x_3$	& $y_3 = p_3$	& $p_{23}$	& $p_{123}$	& $p_{0123}$	& \\
		$x_4$	& $y_4 = p_4$	& $p_{34}$	& $p_{234}$	& $p_{1234}$	& $p_{01234} = p_4(x)$ \\
		$\vdots$ & $\vdots$	&		&		&		& \\
	\end{tabular}
	\caption{Visual representation of Aitken-Neville recursion: $p_{01}$ can
		be determined with $p_0$ and $p_1$, $p_{12}$ can be determined
		with $p_1$ and $p_2$. $p_{012}$ can then be determined with
		$p_{01}$ and $p_{12}$ and so forth.}
\end{table}


\subsection{Newton basis polynomials}
Using the Newton basis polynomials instead of the powers of $1, x^1, x^2, x^3,
\dots, x^m$ further increases the efficiency of the computational scheme for
resolving the collocation problem as it leads to a lower-triangular form of
the system of equations produced by polynomial interpolation.

The Newton basis polynomials are defined as such:
\begin{figure}[H]
	\centering
	\renewcommand{\figurename}{Equations}
	\begin{align*}
		\Pi_0(x) &= 1 \\
		\Pi_1(x) &= (x - x_0) \\
		\Pi_2(x) &= (x - x_0)(x - x_1) \\
		\Pi_3(x) &= (x - x_0)(x - x_1)(x - x_2) \\
		\Pi_4(x) &= (x - x_0)(x - x_1)(x - x_2)(x - x_3) \\
		\vdots \\
		\Pi_k &= (x - x_0)(x - x_1)(x - x_2)(x - x_3) \dots (x - x_{k-1}) \\
		\vdots \\
		\Pi_n &= (x - x_0)(x - x_1)(x - x_2)(x - x_3) \dots (x - x_{n-1})
	\end{align*}
	\caption{Newton basis polynomials for
		$\Pi_k(x) \quad (k = 0,1,2,\dots,n)$}
\end{figure}


Using the Newton basis polynomials for the collocation problem reduces the
system of equations to a single polynomial:
\begin{figure}[H]
	\centering
	\renewcommand{\figurename}{Equations}
	\begin{align}
		p(x) &= a_0 \Pi_0(x) + a_1 \Pi_1(x) + a_2 \Pi_2(x) + a_3 \Pi_3(x)
			+ \dots + a_m \Pi_m(x)
	\end{align}
	\caption{Resulting Newton polynomial to the degree $m$}
\end{figure}

Using more measurements for the collocation problem results in an increase of
the degree of the Newton polynomial:
\begin{figure}[H]
	\centering
	\renewcommand{\figurename}{Equations}
	\begin{align*}
		y_0 = p(x) &= a_0 \\
		y_1 = p(x) &= a_0 + a_1 \Pi_1(x) \\
		y_2 = p(x) &= a_0 + a_1 \Pi_1(x) + a_2 \Pi_2(x) \\
		\vdots \\
		y_k = p(x) &= a_0 + a_1 \Pi_1(x) + a_2 \Pi_2(x) + \dots + a_k \Pi_k(x) \\
		\vdots \\
		y_n = p(x) &= a_0 + a_1 \Pi_1(x) + a_2 \Pi_2(x) + \dots + a_n \Pi_n(x)
	\end{align*}
	\caption{Newton polynomials for $y_k(x) \quad (k = 0,1,2,\dots,n)$
		through application of Aitken-Neville recursion}
\end{figure}


\subsection{Bringing it all together}
Applying the Aitken-Neville recursion to the Newton polynomial leads to
following computational scheme. This is also known as the divided difference:
\begin{align}
	y(x_0, x_1, \dots, x_k) &= \frac{y(x_1, x_2, \dots, x_k) - y(x_0, x_1, \dots, x_{k-1})}
		{x_k - x_0} \quad (k = 0,1,2,\dots,n)
\end{align}

\begin{table}[H]
	\centering
	\begin{tabular}{ r | l }
	$k = 0$ & $y(x_0)$\\
	$k = 1$ & $\frac{y(x_1) - y(x_0)}{(x_1 - x_0)}$\\
	$k = 2$ & $\frac{y(x_1, x_2) - y(x_0, x_1)}{(x_2 - x_0)}$\\
	$k = 3$ & $\frac{y(x_1, x_2, x_3) - y(x_0, x_1, x_2)}{(x_3 - x_0)}$\\
	\end{tabular}
	\caption{Divided difference up to order 3}
\end{table}

\begin{defn}
	Divided Difference: \textit{Dividing the values of two points by the
		step size of said points. This is closely related to the
		derivative.}
\end{defn}

Following simple example demonstrates the computation of the divided
difference in tabular form:
\begin{table}[H]
	\centering
	\begin{tabular}{ l | l l l l }
	x	& y	& $\frac{\Delta y}{\Delta x}$	&		& \\ \hline
	$0$	& $1$	&			&			& \\
	$1$	& $1$	& $\frac{1-1}{1-0} = 0$	&			& \\
	$2$	& $2$	& $\frac{2-1}{2-1} = 1$	& $\frac{1-0}{2-0}$	& \\
	$4$	& $5$	& $\frac{3}{2}$		& $\frac{1}{6}$		& $\frac{-1}{12}$ \\
	\end{tabular}
	\caption{Example for determining the Newton coefficients through divided
		difference. The right-most value of each row is the Newton
		coefficient for given dataset: $a_0 = 1$, $a_1 = 0$, $a_2 =
		\frac{1}{2}$, $a_3 = \frac{-1}{12}$}
\end{table}

This example shows how the computation of the divided difference follows a
simple recursive algorithm. This is the algorithm, which was implemented in
hardware.


\subsection{Limitations}
The considerate algorithm is invariant to the order of the data and the time
deltas between each data point. Meaning the measurement intervals can vary and
the taken measurements can be fed in random order into the algorithm.

For this implementations these properties are limited: The measurements have
to be in a time-coherently order and must have a fixed measurement frequency.
