\section{Introduction}

Bringing a time discrete function, or rather measurements, into a time
continuous domain is often essential for analysis of time discrete data. To do
so, one can either approximate or interpolate. While approximation has
advantages such as being resistant to errors of random nature, interpolation
brings the advantage of being true to the original data. Unlike an
approximation, a interpolation can recall each data point with the
interpolating function.

Again one has a vast variety of interpolation methods at hand, each with their
pros and cons. One such method is the Newton interpolation, a high order
polynomial interpolation. These polynomials can be determined fairly
efficiently using the Aitken-Neville recursion with the divided difference.
This recursion yields the coefficients needed by a Newton polynomial for a
given dataset.

As analysis of measurement data is often performed while data is collected,
the speed of the interpolation algorithm is critical. One way of improving the
speed of algorithm is the implementation in hardware, as highly specific and
optimized hardware is usually faster than a generalized software implementation.

\textit{Outline:} Section 2 shows the mathematics behind the Newton
interplation, the Aitken-Neville recursion, and the divided difference.
Section 3 shows the chosen approach for the implementation in hardware based
on the previously discussed mathematics. Section 4 describes the resulting
hardware implementation while Section 5 discusses the limitations and further
aspects of the deduced hardware. Section 6 concludes the work.
